%=================== HEADER =====================%
% Basic Document Formatting
\documentclass[12pt,a4paper]{article}
\usepackage[T1]{fontenc}
\usepackage[utf8]{inputenc}
\usepackage{geometry}

% Other Preamble
\usepackage{setspace}

\usepackage{fancyhdr}
\setlength{\parindent}{0in}

% Page Formatting
\pagenumbering{arabic} %\pagenumbering{gobble}
\onehalfspacing %doublespacing
\pagestyle{fancy}
%\usepackage{pdfpages}

% Heading Formatting
\headheight 32pt

% Link Formatting
\usepackage{hyperref}
\hypersetup{
	colorlinks,
	allcolors=black
	%citecolor=black,
	%filecolor=black,
	%linkcolor=black,
	%urlcolor=black
}

\usepackage{mdframed}

% Code Formatting
\usepackage{listings}
% \lstset{
%         language=Python,
% 	basicstyle=\footnotesize,
% 	%numbers=left,
% 	stepnumber=1,
% 	showstringspaces=false,
% 	tabsize=1,
% 	breaklines=true,
% 	breakatwhitespace=false
% }

\lstset
{ %Formatting for code  
        language=Python,
	basicstyle=\footnotesize\ttfamily,
	numbers=left,
	%caption={},
	%title={},
	stepnumber=1,
	showstringspaces=false,
	tabsize=1,
	breaklines=true,
	breakatwhitespace=false,
	frame=lines,
	xleftmargin=2em,
	framexleftmargin=1.5em,
	%commentstyle=\color{commentsColor}\ttfamily,
  	%stringstyle=\color{stringColor}\ttfamily,
	%keywordstyle=\color{keywordsColor}\bfseries,
}

% Figures & Drawings
%\usepackage{graphicx, caption}
%\usepackage{animate}
\usepackage{tikz}
\usepackage{float}
\usepackage{pict2e}
\usepackage{subcaption}

% Physics
\usepackage{physics}

% Mathematics
\usepackage{amsmath}
\usepackage{amssymb}
\usepackage{amsthm}
\usepackage{mathtools}
\usepackage{mathrsfs}
\usepackage{upgreek} % More Greek letters

% I don't really know what this is but I don't want to break shit
\usepackage{aliascnt}
\newaliascnt{eqfloat}{equation}
\newfloat{eqfloat}{h}{eqflts}
\floatname{eqfloat}{Equation}
\newcommand*{\ORGeqfloat}{}
\let\ORGeqfloat\eqfloat
\def\eqfloat{%
	\let\ORIGINALcaption\caption
	\def\caption{%
		\addtocounter{equation}{-1}%
		\ORIGINALcaption
	}%
	\ORGeqfloat
}
%}

% Bibliography (Citations) Formatting

%\usepackage{cite}
%\usepackage{caption}

%\usepackage[backend=bibtex,style=verbose-trad2]{biblatex}
%works really really well, but no MLA format
%\bibliographystyle{apsrev4-1}
%\usepackage{biblatex}
%\usepackage[backend=biber]{biblatex}
%\usepackage[backend=biber,style=mla]{biblatex} %Doesn't print all sources for some reason

\usepackage{pgfplots}

\include{../latex-preamble/macros}
\usepackage{mathrsfs}
\bibliography{references.bib}

%============ Document Information ==============%
\newcommand\course{PHY365} %[COURSE INFORMATION!!!]
\newcommand\psetnum{Problem Set 1}  %[PROBLEM SET NUMBER!!!]
\newcommand\yourname{Aditya Rao 1008307761}  %[NAME AND STUDENT NUMBER!!!]
\newcommand{\subject}{\large{\course}}
%================ END OF HEADER =================%

% LaTeX Template by Aditya Rao

\begin{document}
    \title{\psetnum \\ \large{\course}}
	\author{\yourname}
	\date{January 14, 2023}
	\maketitle

	%\pagenumbering{roman}

	\tableofcontents
	
	\newpage
	%So the heading doesn't show up on table of contents page
	%\lhead{\yourname\ \vspace{0.1cm} \\ \course}
	\lhead{\yourname\ \vspace{0.1cm}}
    \chead{\textbf{\subject} \\ \psetnum}
    \rhead{\leftmark}
    
    \newpage
    \section{Question 1}
		\begin{mdframed}
			The three \textit{pauli matricies} are given by:

			\begin{equation*}
				\hat{X} = \begin{bmatrix} 0 & 1 \\ 1 & 0\end{bmatrix}, \ 
				\hat{Y} = \begin{bmatrix} 0 & -i \\ i & 0\end{bmatrix}, \
				\hat{Z} = \begin{bmatrix} 1 & 0 \\ 0 & -1\end{bmatrix}
			\end{equation*}

			Find the eigenvectors, eigenvalues, and diagonal representations of $\hat{X}$, $\hat{Y}$, and $\hat{Z}$.
		\end{mdframed}

		First find eigenvalues by solving the characteristic equation.

		\subsection{For $\hat{X}$}
			Characteristic polynomial: $\lambda^2 - 1$. Therefore, the eigenvalues are $\lambda = \pm 1$.

			In a diagnolized matrix, the eigenvaluesa are the entries along the diagonal. Therefore the diagnolized form of $\hat{X}$ is 

			\begin{equation*}
				\begin{bmatrix} 1 & 0 \\ 0 & -1\end{bmatrix}
			\end{equation*}

			Finally, to find the eigenvectors, we solve the equation $(\hat{X} - \lambda I)\vec{v} = 0$ for each eigenvalue. For $\lambda = 1$, we have.

			\begin{align*}
				\begin{bmatrix}
					-1 & 1 \\ 
					1 & -1
				\end{bmatrix}
				\begin{bmatrix}
					x \\
					y
				\end{bmatrix}
				&= 0 \\
				\implies - x + y = 0 \\
				x - y = 0 \\
				\therefore \vec{v} &= \begin{bmatrix} 1 \\ 1\end{bmatrix}
			\end{align*}

			For $\lambda = -1$, we have

			\begin{align*}
				\begin{bmatrix}
					1 & 1 \\ 
					1 & 1
				\end{bmatrix}
				\begin{bmatrix}
					x \\
					y
				\end{bmatrix}
				&= 0 \\
				\implies x + y &= 0 \\
				x + y &= 0 \\
				\therefore \vec{v} &= \begin{bmatrix} -1 \\ 1\end{bmatrix}
			\end{align*}

		\subsection{For $\hat{Y}$}
			Characteristic polynomial: $\lambda^2 - 1$. Therefore, the eigenvalues are $\lambda = \pm 1$.

			Hence, the diagnolized matrix is the same as that for $\hat{X}$.

			\begin{equation*}
				\begin{bmatrix} 1 & 0 \\ 0 & -1\end{bmatrix}
			\end{equation*}

			Finally, to find the eigenvectors, we solve the equation $(\hat{Y} - \lambda I)\vec{v} = 0$ for each eigenvalue. For $\lambda = 1$, we have.

			\begin{align*}
				\begin{bmatrix}
					-1 & -i \\ 
					i & -1
				\end{bmatrix}
				\begin{bmatrix}
					x \\
					y
				\end{bmatrix}
				&= 0 \\
				\implies - x - iy = 0 \\
				ix - y = 0 \\
				\therefore \vec{v} &= \begin{bmatrix} i \\ 1\end{bmatrix}
			\end{align*}

			For $\lambda = -1$, we have

			\begin{align*}
				\begin{bmatrix}
					1 & -i \\ 
					i & 1
				\end{bmatrix}
				\begin{bmatrix}
					x \\
					y
				\end{bmatrix}
				&= 0 \\
				\implies x - iy &= 0 \\
				ix + y &= 0 \\
				\therefore \vec{v} &= \begin{bmatrix} -i \\ 1\end{bmatrix}
			\end{align*}

		\subsection{For $\hat{Z}$}
			Notice that $\hat{Z}$ is already diagonal. The eigenvalues are the entries along the diagonal, which are $\lambda = \pm 1$.
			
			The eigenvectors are trivially $\vec{v} = \begin{bmatrix} 1 \\ 0\end{bmatrix}$ and $\vec{v} = \begin{bmatrix} 0 \\ 1\end{bmatrix}$ for $\lambda = 1$ and $\lambda = -1$ respectively.

    \section{Question 2}
		\begin{mdframed}
			Show that 
			\begin{equation*}
				\hat{X}\hat{Y} = i\hat{Z}, \ \hat{Y}\hat{Z} = i\hat{X}, \ \hat{Z}\hat{X} = i\hat{Y}
			\end{equation*}

			and hence $\hat{X}\hat{Z}\hat{X} = -\hat{Z}$
		\end{mdframed}
			
		\subsection{Part (a)}
			\begin{align*}
				\hat{X}\hat{Y} &= \begin{bmatrix} 0 & 1 \\ 1 & 0\end{bmatrix}\cdot \begin{bmatrix}0 & -i \\ i & 0\end{bmatrix} \\
				&= \begin{bmatrix} i & 0 \\ 0 & -i\end{bmatrix} \\
				&= i\cdot \begin{bmatrix} 1 & 0 \\ 0 & -1\end{bmatrix} \\
				&= i\hat{Z}
			\end{align*}

			\begin{align*}
				\hat{Y}\hat{Z} &= \begin{bmatrix} 0 & -i \\ i & 0\end{bmatrix}\cdot \begin{bmatrix}1 & 0 \\ 0 & -1\end{bmatrix} \\
				&= \begin{bmatrix} 0 & i \\ i & 0\end{bmatrix} \\
				&= i\cdot \begin{bmatrix} 0 & 1 \\ 0 & 1\end{bmatrix} \\
				&= i\hat{X}
			\end{align*}

			\begin{align*}
				\hat{X}\hat{Y} &= \begin{bmatrix} 0 & 1 \\ 1 & 0\end{bmatrix}\cdot \begin{bmatrix}1 & 0 \\ 0 & -1\end{bmatrix} \\
				&= \begin{bmatrix} 0 & -1 \\ 1 & 0\end{bmatrix} \\
				&= i\cdot \begin{bmatrix} 0 & -i \\ i & 0\end{bmatrix} \\
				&= i\hat{Y}
			\end{align*}

		\subsection{Part (b)}
			Substiute $\hat{Z}\hat{X} = i\hat{Y}$.
			\begin{align*}
				\hat{X}\hat{Z}\hat{X} &= \hat{X}(i\hat{Y}) \\
				&= i\hat{X}\hat{Y} \\
				&\text{Substiute} \ \hat{X}\hat{Y} = i\hat{Z} \\
				&= i(i\hat{Z}) \\
				&= -\hat{Z}
			\end{align*}
	
    \section{Question 3}
		\begin{mdframed}
			A \textit{unitary} matrix is a matrix whose inverse is equal to its Hermitian adjoint. (i.e take the transpose and the complex conjugate). Show that 
			\begin{equation*}
				\hat{U} = \begin{bmatrix} a & b \\ -b^* & a^* \end{bmatrix}
			\end{equation*}

			Where ($|a|^2 + |b|^2 = 1$) is unitary.
		\end{mdframed}

		Find the Hermitian adjoint of $\hat{U}$. I.e we find $\hat{U}^{\dagger} = (\hat{U}^T)^*$
		
		\begin{align*}
			\hat{U}^{\dagger} &= (\hat{U}^T)^* \\
			&= \left(\begin{bmatrix} a & b \\ -b^* & a^* \end{bmatrix}^*\right)^T = \begin{bmatrix} a & -b^* \\ b & a^* \end{bmatrix}^* \\
			&= \begin{bmatrix} a^* & -b \\ b^* & a \end{bmatrix}
		\end{align*}

		Now, for a matrix to be unitary, it must mean that $\hat{U}\hat{U}^{\dagger} = I$. We can check this.

		\begin{align*}
			\begin{bmatrix} a & b \\ -b^* & a^* \end{bmatrix}\cdot\begin{bmatrix} a^* & -b \\ b^* & a \end{bmatrix} &= \begin{bmatrix} aa^* + bb^* & -ab - ba \\ -b^*a^* + a^*b^* & b^*b + a^*a \end{bmatrix} \\
			&= \begin{bmatrix} |a|^2 + |b|^2 & 0 \\ 0 & |a|^2 + |b|^2 \end{bmatrix} \\
			&= \begin{bmatrix} 1 & 0 \\ 0 & 1 \end{bmatrix}
		\end{align*}

		Therefore, $\hat{U}$ is unitary.
	
	\newpage
	\section{References}
		\printbibliography[heading=none]

\end{document}